\section{Next steps}

\noindent In order for the plan described in this document to be put into action, a number of steps are required:

\subsection{Establish release contents}

\noindent An initial set of extensions/changes were presented by each team member for testbed 1 at the 7th of December 2020 meeting. These lists are a starting point for planning the contents of the scheduled releases. Team members are asked to complete the information about each item on their proposed list by providing the following details:
\begin{itemize}
\item \textbf{Priority:}
Prioritise the list of items in consultation with the technical coordinator and project manager;
\item \textbf{Effort:}
Estimate the effort required to develop/modify the software and identify the individuals/groups involved;
\item \textbf{Dependencies:}
Identify the dependencies on other software packages (internal or external software products);
\end{itemize}

\noindent \textbf{Note:} An item that needs to be added to every team member list is the use of the infrastructure tools required to support the release procedure. The training in the use of the infrastructure tools is the responsibility of the technical coordinator.\\

\noindent Once such information is available from all the team member contributing to the software releases, it will be possible to map them on to the foreseen release schedule. This mapping will have to take into account dependencies between team members. Such a plan will be useful to help organise the testing and validation of the application. The input of the team members is important for the release planning procedure. It is expected that, the work of the validation will be affected by making interventions more often but with reduced intensity.\\

\noindent It is hoped that the overall release schedule including each team members extensions/changes can be compiled during December 2020. This release schedule will be updated after each release using the output of the software release work-plan coordination.\\

\noindent \textbf{Note:} Planning is never perfect – team members may under estimate the effort required for a particular item or a new release of an external software package may not be delivered on time. In such cases priority will be given to respecting the release schedule. In other words, the scheduled date for the saman software releases will not change but it may contain less changes/extensions than originally planned.

\subsection{Develop testplans}

\noindent An important point to clarify the status of the software and simplify integration, is the development of testplans for all major components of the software. The technical coordinator should develop testplans for the software based on a standard project template. The technical coordinator should also develop an overall testplan corresponding to the integration tests for the application and external packages.

\subsection{Deploy infrastructure}

\noindent In parallel to establishing the contents of each software release, the technical coordinator can start deploying the infrastructure necessary to support the release procedure. This will require the cooperation of the team members developing software that shall be expected to start making use of the various tools and facilities as they become available. In particular, packages that identifying the unit tests that can be performed before delivering the software to the \texttt{dev} branch for integration.

\subsection{Gather resources for a development deployment}

\noindent In parallel to the above steps, the technical coordinator can contact the team members to establish a distributed development plan. With the experience already gained from the releases, the technical coordinator is in a position to give details of the minimum number of pull requests required at each branch to represent a useful development deployment. 