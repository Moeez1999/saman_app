\section{Introduction}

\subsection{Objectives of this document}

\noindent This document describes a proposed policy for organising the production and deployment of successive software releases within the saman project.\\

\noindent The overall saman project schedule includes 2 testbeds and 8 releases over a 6 month period. For each testbed or release, a software release is conducted. Such software releases integrate the latest developments from each of the feature branches from the team developers with the \texttt{dev} branch and furthermore an integration is also done with the \texttt{master} branch, i.e., the production environment.\\

\noindent The coordination of the software developments in each feature branch and their integration is an important task that requires considerable effort.\\

\noindent On the \textbf{1st of February 2021} the saman project should have a “working-product” in production and model in which integration happens more frequently with incremental steps to minimise the effort and time involved and permits user feedback to be incorporated more quickly.\\

\noindent This document presents a proposed policy for iterative software releases including milestones, time-scales and supporting techniques and tools to be employed.\\

\noindent The policy outlined in this document is presented as a detailed implementation plan for each software release associated with the saman project. Each team member is recommended to read and supply feedback to the contents in the document.

\subsection{Application area}

\noindent This document concerns the technical coordination of the software releases for the saman project. It has a consequence on the manner in which the software of the project is produced, integrated and tested by future users and stakeholders.

\subsection{Applicable and reference resources}

\textbf{Reference resources}
\begin{itemize}
    \item saman Central Github Repository - \texttt{https://github.com/seemir/saman};
    \item Official Slack workplace for the saman project\newline - \texttt{https://app.slack.com/client/T01A0N5PDAL};
\end{itemize}

\subsection{Document amendment procedure}

\noindent Readers are invited to send comments on the saman slack project to the author who will consolidate all feedback and produce updated versions of the document when significant changes are made.

\subsection{Terminology}

\noindent \textbf{Definitions:}
\begin{itemize}
    \item saman - Project for the introduction of a Customer-driven logistics platform primary to be used in the transportation of goods and services market in Pakistan;
\end{itemize}

\subsubsection{Glossary}
\begin{itemize}
    \item Github - Version control (also known as revision control, source control, or source code management) systems responsible for managing changes to the saman app;
    \item Slack - Business communication platform used by the saman team to communicate changes, news, updates and any questions or feedback;
\end{itemize}

\subsection{Role of the technical coordinator}

\noindent It is the role of the technical coordinator to organise and oversee the software release planning of the project in consultation with the project management and team members. These tasks include:
\begin{itemize}
\item	Proposing a software release policy and procedure to drive the development of each software release;
\item	Defining a software release calendar including release dates and content outlines;
\item	Organising a forum so that technical issues involving multiple issues can be addressed;
\item	Working with the project managers to ensure work progress is harmonized with overall project plans, including input from users and stakeholders;
\item	Establishing a basic software infrastructure and toolset to facilitate software development, continuous integration and testing;
\item	Tracking the progress of the work to ensure the software release schedule progresses as smoothly as possible.
\end{itemize}